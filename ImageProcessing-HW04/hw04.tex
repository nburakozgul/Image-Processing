\documentclass[12pt]{article}


\usepackage{amsmath} 

\title{Image Processing HW04}
\author{Necati Burak Ozgul}
\date{May 19 2017}
	
\begin{document}
\maketitle

\section{HW04 Part1 }


Kosinüsün tanimindan asagidaki fonksiyona ulasiyoruz.

\displaystyle \cos x = \sum_{n \mathop = 0}^\infty \left({-1}\right)^n \frac {x^{2n}}{\left({2n}\right)!}

\displaystyle D_x \left({\cos x}\right) = \sum_{n \mathop = 1}^\infty \left({-1}\right)^n 2n \frac {x^{2n - 1} }{\left({2n}\right)!}
 
\displaystyle = \sum_{n \mathop = 1}^\infty \left({-1}\right)^n \frac {x^{2n - 1} }{\left({2n - 1}\right)!}

\displaystyle = \sum_{n \mathop = 0}^\infty \left({-1}\right)^{n+1} \frac {x^{2n + 1} }{\left({2n + 1}\right)!}

\displaystyle = - \sum_{n \mathop = 0}^\infty \left({-1}\right)^n \frac {x^{2n + 1} }{\left({2n + 1}\right)!}

Seklinde derivative oldugunu ispatlariz.

\section{HW04 Part2 }

Hocam part2'yi yapamadim ancak yaparken kullanmak istedigim yontemi aciklamak istiyorum. Image uzerinde edge detection yaptiktan sonra edge noktalarina tek boyutta fft uyguladiktan sonra bunlari vektore koyarak feature olarak kullanabiliriz. 

\end{document}
